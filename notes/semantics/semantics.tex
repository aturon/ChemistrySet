\documentclass[10pt]{article}

\usepackage[top=1.25in,bottom=1.25in,left=1.5in,right=1.5in]{geometry}
\usepackage{mathpazo}
\usepackage{mathpartir}

\newcommand{\ra}{\rightarrow}
\newcommand{\gives}{\vdash}
\newcommand{\proves}{\vdash}
\newcommand{\GA}{\ |\ }
\newcommand{\kw}[1]{\textsf{#1}}
\newcommand{\ty}[1]{\textrm{#1}\ }
\newcommand{\IO}{\mathbb{IO}\ }
\newcommand{\R}{\mathbb{R}\ }
\newcommand{\M}{\textit{M}\ }
\newcommand{\IOb}{\mathbb{IO}}
\newcommand{\Rb}{\mathbb{R}}
\newcommand{\Mb}{\textit{M}}
\newcommand{\bind}{\mbox{\ \scriptsize$>\!\!>\!=$\ }}
\newcommand{\letrec}[2]{\kw{letrec}\ #1\ \kw{in}\ #2}
\newcommand{\LET}[4]{\kw{let}_{#1}\ #2 = #3\ \kw{in}\ #4}

\title{Reagent semantics}
\author{Aaron Turon}

\begin{document}

\maketitle

\section{Grammar and type rules}

\[
\begin{array}{rcl}
e &::=& 
      x \GA \lambda x. e \GA e(e) \GA (e_1, \dots, e_n) \GA \pi_i(e) \GA 
      \kw{return}_\Mb(e) \GA e \bind e \\
  &|& \kw{ref}(e) \GA \kw{read}(e) \GA \kw{cas}(e,e,e) \GA 
      \kw{chan} \GA \kw{send}(e,e) \GA 
      \kw{block} \GA \kw{retry} \GA e + e  \\
  &|& \kw{react}(e) \GA \kw{dissolve}(e) \GA \kw{spawn}(e) \\
%% e &::=& 
%%       v \GA \LET{\Mb}{x}{c}{e} \\
\tau &::=& 
      \tau \ra \tau \GA (\tau_1, \dots, \tau_n) 
  \GA \ty{ref}\tau \GA \ty{ep}(\tau, \tau') \GA \M\tau 
%\Mb &::=& \Rb \GA \IOb
\end{array}
\]
\[
\infer
  {}
  {\Gamma, x:\tau \gives x :\tau}
\qquad
\infer
  {\Gamma, x:\tau \gives e: \tau'}
  {\lambda x.e : \tau \ra \tau'}
\qquad
\infer
  {\Gamma \gives e_1 : \tau_2 \ra \tau \\
   \Gamma \gives e_2 : \tau_2}
  {\Gamma \gives e_1(e_2) : \tau}
\]
\[
\infer
  {\Gamma \gives e_i : \tau_i}
  {\Gamma \gives (e_1, \dots, e_n) : (\tau_1, \dots, \tau_n)}
\qquad
\infer
  {}
  {\Gamma \gives \pi_i : (\tau_1, \dots, \tau_n) \ra \tau_i}
\]
\[
\infer
  {\Gamma \gives e : \tau}
  {\Gamma \gives \kw{return}_\Mb(e) : \M\tau}
\qquad
\infer
  {\Gamma \gives e : \IO\tau \\\\
   \Gamma \gives e_k: \tau \ra \IO\tau'}
  {\Gamma \gives e \bind e_k : \IO\tau'}
\qquad
\infer
  {\Gamma \gives e : \R\tau \\\\
   \Gamma \gives e_k: \tau \ra \IO(\R\tau')}
  {\Gamma \gives e \bind e_k : \R\tau'}
\]
\[
\infer
  {\Gamma \gives e : \tau}
  {\Gamma \gives \kw{ref}(e) : \IO(\ty{ref}\tau)}
\qquad
\infer
  {\Gamma \gives e : \ty{ref}\tau}
  {\Gamma \gives \kw{read}(e) : \R\tau}
\qquad
\infer
  {\Gamma \gives e : \ty{ref}\tau \\
   \Gamma \gives e_{\rm old} : \tau \\
   \Gamma \gives e_{\rm new} : \tau}
  {\Gamma \gives \kw{cas}(e, e_{\rm old}, e_{\rm new}) : \R ()}
\]
\[
\infer
  {}
  {\Gamma \gives \kw{chan} : \IO(\ty{ep}(\tau, \tau'),\ \ty{ep}(\tau', \tau))}
\qquad
\infer
  {\Gamma \gives e: \ty{ep}(\tau, \tau') \\
   \Gamma \gives e_{\rm msg}: \tau}
  {\Gamma \gives \kw{send}(e, e_{\rm msg}) : \R\tau'}
\]
\[
\infer
  {}
  {\Gamma \gives \kw{block} : \R\tau}
\qquad
\infer
  {}
  {\Gamma \gives \kw{retry} : \R\tau}
\qquad
\infer
  {\Gamma \gives e_i : \R\tau}
  {\Gamma \gives e_1 + e_2 : \R\tau}
\]
\[
\infer
  {\Gamma \gives e: \R\tau}
  {\Gamma \gives \kw{react}(e): \IO\tau}
\qquad
\infer
  {\Gamma \gives e: \R()}
  {\Gamma \gives \kw{dissolve}(e): \IO()}
\qquad
\infer
  {\Gamma \gives e: \IO()}
  {\Gamma \gives \kw{spawn}(e): \IO()}
\]

\section{Evaluation contexts}

\newcommand{\EC}{\mathcal{E}}
\newcommand{\PC}{\mathcal{P}}

%% \begin{eqnarray*}
%%   \EC &::=& [\cdot] \GA 
%% \end{eqnarray*}

\section{Core reduction rules}

\newcommand{\step}[1]{\stackrel{#1}{\longrightarrow}}

\begin{eqnarray*}
  \kw{return}_\IO(v) \bind v_k &\step{}& v_k(v) \\
  \kw{return}_\IO(\kw{return}_\R(v)) \bind v_k &\step{}& v_k(v) \\
  (\lambda x. e)(v) &\step{}& e\{v/x\}
\end{eqnarray*}

%% \begin{eqnarray*}
  
%% \end{eqnarray*}

\end{document}

Some nice ideas:

- Use fractional permissions model to capture isolation.

- Use process algebraic LTS to model communication.

- A successful reaction is a group of reagents taht tau-step*s to
`return`s
